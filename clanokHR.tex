
% Metódy inžinierskej práce
\documentclass[10pt,twoside,english,a4paper]{article}

\usepackage[english]{babel}
%\usepackage[T1]{fontenc}
\usepackage[IL2]{fontenc} % lepšia sadzba písmena Ľ než v T1
\usepackage[utf8]{inputenc}
\usepackage{graphicx}
\usepackage{url} % príkaz \url na formátovanie URL
\usepackage{hyperref} % odkazy v texte budú aktívne (pri niektorých triedach dokumentov spôsobuje posun textu)

\usepackage{cite}
%\usepackage{times}

\pagestyle{headings}

\title{Positive impact of gaming on english skills as a second language\thanks{Semestrálny projekt v predmete Metódy inžinierskej práce, ak. rok 2022/23, vedenie: Vladimír Mlynarovič}} % meno a priezvisko vyučujúceho na cvičeniach

\author{Samuel Hronec\\[2pt]
	{\small Slovak technical university in Bratislave}\\
	{\small Faculty of informatics and information technologies}\\
	{\small \texttt{xhronec@stuba.sk}}
	}

\date{\small 6. november 2022} % upravte



\begin{document}

\maketitle

\begin{abstract}
This article deals with gaming, i.e. playing games, whether on a computer, console or phone, and its impact on the level of English as a secondary language among young people. Often these games are not available in the native language of the players and so they are forced to play them in a foreign language, of which English is the most common choice. In addition, some games require online communication with teammates, so there is also oral input of English, not just written. The article mainly concerns children in primary schools and teenagers in high schools, i.e. children with approximately 2 hours of English as part of their lessons. The difference is shown in the results of written tests, as well as the ability to converse in English. We will also deal with the differences between the genders and how it affects the given factor. By examining computer game playing and its relationship to English as a secondary language, much information is revealed and the importance of English literacy in the modern world and the impact of gaming and its potential benefits in terms of English language skills.
\end{abstract}



\section{Introduction}

It is no secret that video games are extremely popular among people of all ages. In fact, recent studies show that the average person spends over six hours per week playing video games. This is especially true for young people, who often spend even more time gaming than they do watching television, reading books or even using the internet and browsing social media. Given the popularity of gaming, it is not surprising that there has been growing interest in the impact of gaming on English skills, especially those of school age with English as their second language.
There is evidence to suggest that gaming can have a positive impact on English skills. For example, one study from Denmark \cite{Denmark_study} showed, that gaming with both written and spoken input and gaming with only written input have significant are significantly related to English skills and vocabulary scores. Other study from Taiwan \cite{Taiwan_study} has also found that learners who learned english through drill and game based exercises performed better in terms of pronunciation, than their peer who performed only drill based exercises. Furthermore, as games are often set in real-world or fantasy settings, they can also help to build knowledge about other cultures which may be beneficial for those who are learning English as a second language.



The positive impact of gaming English skills as second language is here~\ref{positives}.
The differences between genders in English skills, related to gaming is here~\ref{genders}.




\section{Positive impact and ways to improve} \label{positives}
There are a number of factors that contribute to the positive impact of gaming on English skills. Firstly, games provide a fun and entertaining way to learn. They are also interactive and engaging, which encourages players to keep coming back for more. Secondly, many games now offer the option to play in English or with English subtitles, which means that learners can immerse themselves in the language while they play. Finally, some games also require players to use specific strategies and problem-solving skills, which can help to develop thinking and reasoning skills.

There are lots of different ways to use gaming to improve your English skills. One option is to find an online game that you can play with other English-speakers. This is a great way to practise your conversation skills and learn new vocabulary in a fun and relaxed environment. Alternatively, you could try playing a single-player games such as Minecraft. These types of games can still be very useful for learning English, as they often contain lots of instructions and text that you can read and practise using.

Plus, students can also use gaming to improve their listening skills. Many games now come with audio options, so you can listen to the characters speaking and get used to hearing different accents and pronunciations. This can be a great way to improve your comprehension and learn new words too. 

All of these factors combine to make gaming a great way to learn English as a second language. 

\section{Differences between genders} \label{genders}
There is a growing body of research that suggests, that there are differences between genders when it comes to English skills and gaming \cite{Gender_diff}. These differences can be traced back to the different ways they approach gaming overall. The research has shown that girls tend to focus more on social aspects of gaming, while boys tend to focus more on competition and individual achievement. This difference may be due to the fact that men tend to play more action-oriented games, while women tend to play more social games. This difference in perspective leads to different linguistic behaviors in each gender. 

It has also found that men are more likely than women to use English as a second language in online gaming communities. This is likely due to the fact that men are more likely to play online games than women.

Girl gamers are more likely to use language that builds relationships and creates a sense of community. They use phrases like "let's team up" or "we should group together" more often than boys do. They're also more likely to use emotive language, expressing themselves emotionally in order to create bonds with other players.

Boys, on the other hand, are more likely to use language that is direct and to the point. They're more likely to give orders or issue challenges than girls are. And they're more likely to use taboo and vulgar words than girls are.

Too add, according to the Denmark study \cite{Denmark_study}, boys overall gamed significantly more than their girl peers. The boys gamed on average 235 minutes a week, in contrast with girls, that gamed on average 47 minutes a week. Plus, according to the study, the most preferred free time activity for girls was listening to music with 115 minutes a week, followed by watching television with 69 minutes. On the other hand, the preferred activity for boys was gaming with 235 minutes a week followed by watching television with 120 minutes a week and listening to music with 78 minutes a week.






%\acknowledgement{Ak niekomu chcete poďakovať\ldots}


% týmto sa generuje zoznam literatúry z obsahu súboru literatura.bib podľa toho, na čo sa v článku odkazujete
\bibliography{literatura}
\bibliographystyle{plain} % prípadne alpha, abbrv alebo hociktorý iný
\end{document}
                    